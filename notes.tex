\documentclass{article}
\usepackage[utf8]{inputenc}
\usepackage[margin=0.75in]{geometry} % lots more margin
\pagenumbering{gobble} % ignore page numbers

\usepackage{titling}
\setlength{\droptitle}{-0.75in}

\setlength{\parindent}{0cm}

\usepackage{enumitem}
\usepackage{graphicx}
\usepackage{amsmath}
\usepackage{amsfonts}
\usepackage{hyperref} % for nice looking urls
\usepackage{booktabs} % for making tables
\usepackage{amssymb}
\usepackage{listings}
\usepackage{graphicx}
\usepackage{caption}
\usepackage{subfigure}
\usepackage{multicol}

\usepackage{titlesec}

\titleformat*{\section}{\large\bfseries}

\begin{document}

\begin{multicols*}{2}

\section*{MATH342 Review Notes}

By Jack `jryzkns' Zhou

\section{The division Algorithm}

If $a, b \in \mathbb{Z}$ with $b > 0$, then $\exists ! (q, r) \in \mathbb{Z}$\footnote{The symbol $\exists!$ indicates that the existence is unique} s.t.

\[ a = bq + r; \;\;\; 0 \leq r < b \]

\section{Divisibility and Primes}

If $a, b \in \mathbb{Z}$ with $a \neq 0$, we say that $a$ \textit{divides} $b$ if $\exists c \in \mathbb{Z}$ s.t. $b = ac$. We write this relationship as $ a \mid b$. If $a$ \textit{does not divide} $b$, we write $a \nmid b$.

Below are some properties of divisibility, using $a,b,c \in \mathbb{Z}$ and $m,n \in \mathbb{Z}$:

\begin{itemize}
\item $a \mid b \wedge b \mid c \rightarrow a | c$
\item $c \mid a \wedge c \mid b \rightarrow c | (ma + nb)$
\end{itemize}

An integer $n$ is said to have \textit{odd} parity when $n \nmid 2$, and otherwise \textit{even} parity when $n \mid 2$. Often times an odd number can be written in the form $n = 2k + 1$ for some integer $k$, and $n = 2k$ for even numbers.

A \textit{prime} is an integer greater than 1 thatis divisible by no positive integers other than $1$ and itself. Otherwise, a number is said to be a \textit{composite}. Note that by this definition. The number $2$ is a prime. Often times there will be indications to exclude $2$ from primes by specifying odd primes.

Here are some facts about prime numbers, where $n \in \mathbb{Z}$:

\begin{itemize}
\item when $n > 1$, $p \mid n$ for some prime $p \leq n$
\item There are infinitely many primes
\item If $n$ is composite, then $p \mid n$ for some prime $p \leq \sqrt{n}$
\end{itemize}

The function $\pi(x)$ where $x$ is a positive real number denotes the number of primes not exceeding $x$.

\subsubsection*{Dirichlet's Theorem in Arithmetic Progressons}

Suppose that $a, b \in \mathbb{N}$ where $(a, b) = 1$. Then the arithmetic progression $an+b, n \in \mathbb{N}$ contains infinitely many primes.

\section{Greatest Common Divisors}

The \textit{greatest common divisor} (GCD) of $a, b \in \mathbb{Z}$ is the largest divisor $d$ such that $d \mid a$ and $d \mid b$. The GCD of $a$ and $b$ are often written as $\gcd(a, b)$ or $(a, b)$.

One way to describe the GCD is that a positive integer $d$ is a GCD iff:

\begin{itemize}
\item $d \mid a$ and $d \mid b$
\item if $c \in \mathbb{Z}$ s.t. $c \mid a$ and $c \mid b$, then $c \mid d$
\end{itemize}

Some facts about GCD's:

\begin{itemize}
\item Two integers $a, b \in \mathbb{Z}$ are said to be \textit{relatively prime} if $(a, b) = 1$
\item Suppose $d = ( a, b )$, then $(\frac{a}{d},\frac{b}{d}) = 1$
\item A fraction $\frac{p}{q}$ is in lowest terms when $(p, q) = 1$
\item suppose $(a, b) = 1$ and $a | bc$, then $a | c$
\item The notion of GCD's applies to multiple values too, suppose $a_1,a_2, \cdots, a_n \in \mathbb{Z}$, then \[(a_1, a_2, \cdots, a_n) = (a_1, a_2, \cdots, (a_{n-1}, a_n))\]
Note that simply with the above definition, $(a_1, a_2, \cdots, a_n) = 1$ means that these numbers are \textit{mutually relatively prime}. A stronger statement is \textit{pairwise relatively prime}, where for any pair of the numbers $a_i$ and $a_j$ with $i \neq j$, $(a_i, a_j) = 1$.
\end{itemize}

\subsubsection*{Bezout's Theorem}

If $a, b \in \mathbb{Z}$, then $\exists m,n \in \mathbb{Z}$ s.t. \[ma + nb = (a, b)\] Where $m,n$ are denoted the bezout coefficients to $a, b$. Furthermore, $(a,b) = 1 \Leftrightarrow ma +nb = 1$.

The set of linear combinations of $a,b$ is the set of integer multiples of $(a, b)$.

\subsubsection*{Euclidean Algorithm (+Backtracking)}

The Euclidean Algorithm computes $(a,b)$. It proceeds by heavily using a property of GCD's:

Let $b,q,r \in \mathbb{Z}$, \[(bq + r, b) = (b, r)\]

Suppose $a = bq+r$ (by the division algorithm), we have $(a,b) = (b, r)$. As $0 \leq r < b$, the RHS will always be in lower terms: each time we apply this property, we will get smaller computations to carry out until a base case of $(b, 0)$ is reached, in which case $(b, 0) = b$.

Once the series of division algorithms are applied, the results can be used in reverse with substitution to find the bezout coefficients.

As a side note, suppose we have $f_{n+1}, f_{n+2}$ be successive terms of the Fibonacci Sequence with $n > 1$, then the Euclidean algorithm takes exactly $n$ divisions to show that $(f_{n+1}, f_{n+2})= 1$.

\section{Continued Fraction Expansions}

Given the sequence $a_0, a_1, a_2, \cdots$ (may be infinite), a continued fraction is a fraction of the following form

\[a_0 + \frac{1}{a_1 + \frac{1}{a_2 + \cdots}}\]

A fraction of this form can also be written as $[a_0,a_1, a_2, \cdots]$. If the sequence is infinite and has repeating elements, we denote one instance of the repeating values with a line drawn over it.

The numbers $a_1, a_2, \cdots, a_n$ are called \textit{partial quotients} of the continued fraction, and if all $a_i$'s are integers, the continued fraction is said to be a \textit{simple} continued fraction. We are only concerned with \textit{simple} continued fractions and unless otherwise stated, all continued fractions under discussion are simple.

\subsection*{Finite Continued Fractions}

For a rational number that can be expressed as $\frac{p}{q}$, where $p,q \in \mathbb{Z}$ and $p > q$. Its continued fraction expansion is finite. The way to generate the expansion is to consider the division algorithms applies when computing $(p, q)$. For each row, it will be of the form $a = q\cdot b + r$. We apply the following transformation:

\[a = bq + r \Rightarrow \frac{a}{b} = q + \frac{1}{\frac{b}{r}}\]

The $\frac{b}{r}$ term will correspond to the LHS on the next row, recall that the next row would be computing $b$ divided by $r$, and thus a substitution can be made. We can continuously apply these substitutions until a continued fraction is formed.

\subsection*{Infinite Continued Fractions}

For an irrational number $n$, its continued fraction expansion is computed in the following manner:

\[
\begin{aligned}
&\;\;\; \alpha_0 = n \\
a_i = [\alpha_i] &\;\;\; \alpha_{i+1} = (\alpha_i - a_i)^{-1}
\end{aligned}
\]

It's good to note that $a_i$ is the integral component to $\alpha_i$, and that $\alpha_i - a_i$ is the fractional component to $\alpha_i$.

\subsection*{Convergents}

Given a continued fraction \[C = [a_0;a_1,a_2,\cdots,a_n]\] and $C_k = [a_0;a_1,a_2,\cdots,a_k]$ with $0 < k \leq n$, $C_k$ is called the $k$th convergent of $C$. Given $C$, we can compute all of the convergents of $C$ as follows:

\[
\begin{aligned}
& p_0 = a_0 & q_0 &= 1 &\\
& p_1 = a_0a_1+1 & q_1 &= a_1 & C_1 = \frac{p_1}{q_1}\\
& p_k = a_kp_{k-1}+p_{k-2}& q_k &= a_kq_{k-1}+q_{k-2} & C_k = \frac{p_k}{q_k}
\end{aligned}
\]

\section{Linear Diophantine Equations}

A linear diophantine equation in two variables is an equation of the follow form \[ax+by=c\] where $a,b,c \in \mathbb{Z}$ and an integer solution $(x,y)$ is sought for. In general, an equation that is linear in the coefficients to the polynomial powers would be considered a diophantine equation. We will focus on linear diophantine equations in two variables. Much of the results we find here applies to diophantine equations of more than two variables.

Let $d = (a, b)$. If $d \nmid c$, then the diophantine equation in question has no integral solutions. If $d \mid c$, then there are infinitely many solutions (granted that there are no restrictions on the solution space). Indeed, once we have an initial solution $(x_0, y_0)$, the rest of the solutions will all be of the form

\[x = x_0 + \frac{b}{d}n,\;\;\;y = y_0 - \frac{a}{d}n \]

for any integer $n$. Notice the minus sign on $y$.

\subsection*{Systems of Linear Diophantine Equations}

When there is a system of linear diophantine equations (most commonly two equations in 3 variables), the ideal approach is to substitute one equation into another, to reduce the system into a single equation. From there, solve as intended and substitute back for a full solution.

\section{The Fundamental Theorem of Arithmetic}

Every positive integer greater than 1 is \textit{uniquely} expressible as a product of primes, with the prime factors in non decreasing order: \[n = \prod_{i} p_i^{a_i}\]

The sequence of $a_i$'s denote the exponents to each of the prime factors. If a prime factor $p_i \nmid =n$, then $a_i = 0$.

Using this prime factorization form, we can describe the GCD and LCM (least common multiple) as follows, letting $a = \prod_{i}p_i^{a_i}$ and $b = \prod_{i}p_i^{b_i}$:

\[(a,b) = \prod_{i}p_i^{\min(a_i,b_i)}\;\;[a,b] = \prod_{i}p_i^{\max(a_i, b_i)}\]

As $\max(a_i, b_i) + \min(a_i, b_i) = a_i + b_i$, we can see from above that $(a,b) \cdot [a,b] = ab$

\section{Congruences}

Let $m$ be a positive integer. If $a,b \in \mathbb{Z}$, we say that $a$ is \textit{congruent} to $b$ modulo $m$ if $m \mid (a - b)$. Furthermore, we can write $a \equiv b \pmod{m}$ iff $a = b + km$ for some integer $k$. Here are some properties about arithmetic in modulo $m$:

\begin{itemize}
\item $a+c \equiv b + c \pmod{m}$
\item $a - c \equiv b - c \pmod{m}$
\item $ab \equiv bc \pmod{m}$
\end{itemize}

The congruence relation over a modulus forms an \textit{equivalence class}, which satisfies the following properties:

\begin{itemize}
\item Reflexitivity: $a \equiv a \pmod{m}$
\item Symmetricity: $a \equiv b \pmod{m} \Leftrightarrow b \equiv a \pmod{m}$
\item Transitivity: $a \equiv b \pmod{m} \wedge b \equiv c \pmod{m} \Rightarrow a \equiv c \pmod{m}$
\end{itemize}

A number in a modulo number system itself is referred to as a residue. If $a, b$ are congrument mod $m$, then their residues $a\pmod{m}$ and $b\pmod{m}$ have to be the same as well.

A complete system of residues modulo $m$ is a set of integers such that every integer is congruent to exactly one integer of the set. Of which, the least non-negative residues modulo $m$ is the set \[\{0,1,\cdots,m -1\}\] When $m$ is odd, the absolute least residues modulo $m$ is the set \[\left\{-\frac{m-1}{2}, -\frac{m-3}{2}, \cdots, -1, 0, 1, \frac{m-3}{2}, \frac{m-1}{2}\right\}\]

If $a,b,c,m \in \mathbb{Z}^+$ with $d = (c, m)$ and $ac \equiv bc \pmod{m}$, then $a \equiv b \pmod{\frac{m}{d}}$. In the case that $(c, m) = 1$, we simply have $a \equiv b \pmod{m}$.

If we have $a \equiv b \pmod{m_i}$ for several values of $i$ and each $m_i$ are pairwise coprime, then $a \equiv b \pmod{\prod_im_i}$.

\newpage

\section{Infinimality of primes of the form $4k+3$}

Before the actual proof can take place, we note that primes can only either be of the form $4k+1$ or $4k+3$, which are the odd numbers in mod 4.

Also we note that if two numbers $a, b$ are both of the form $4k +1$, then $ab$ will also be of the form $4k + 1$ (proof omitted for brevity).

To begin the actual proof, suppose we have a finite amount of primes of the form $4k+3$, Namely, all in the sequence $\{p_i\}$ starting with $p_0 = 3$. We construct the following: \[Q = 4\prod_{i=1}{p_i} + 3\]Note that $p_0 = 3$ is omitted from the expression.

Clearly, $Q$ is a composite value of the form $4k+3$ by construction. It needs to have a factor that is of the form $4k+3$, as we have established that multiplication between numbers of the form $4k+1$ is closed. Therefore, for some $i$, $p_i \mid Q$. However, none of the $p_i$'s can divide $Q$ by construction. Therefore, $Q$ is another prime of the form $4k+3$, a contradiction to the assumption that there are only a finite amount of them.

\end{multicols*}
\end{document}
