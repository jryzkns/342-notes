\documentclass{article}
\usepackage[utf8]{inputenc}
\usepackage[margin=0.75in]{geometry} % lots more margin
\pagenumbering{gobble} % ignore page numbers

\usepackage{titling}
\setlength{\droptitle}{-0.75in}

\setlength{\parindent}{0cm}

\usepackage{enumitem}
\usepackage{graphicx}
\usepackage{amsmath}
\usepackage{amsfonts}
\usepackage{hyperref} % for nice looking urls
\usepackage{booktabs} % for making tables
\usepackage{amssymb}
\usepackage{listings}
\usepackage{graphicx}
\usepackage{caption}
\usepackage{subfigure}
\usepackage{multicol}

\usepackage{titlesec}

\titleformat*{\section}{\large\bfseries}

\begin{document}

\begin{multicols*}{2}

\section*{MATH342 Review Notes}

By Jack `jryzkns' Zhou

\section{The division Algorithm}

If $a, b \in \mathbb{Z}$ with $b > 0$, then $\exists ! (q, r) \in \mathbb{Z}$\footnote{The symbol $\exists!$ indicates that the existence is unique} s.t.

\[ a = bq + r; \;\;\; 0 \leq r < b \]

\section{Divisibility and Primes}

If $a, b \in \mathbb{Z}$ with $a \neq 0$, we say that $a$ \textit{divides} $b$ if $\exists c \in \mathbb{Z}$ s.t. $b = ac$. We write this relationship as $ a \mid b$. If $a$ \textit{does not divide} $b$, we write $a \nmid b$.

Below are some properties of divisibility, using $a,b,c \in \mathbb{Z}$ and $m,n \in \mathbb{Z}$:

\begin{itemize}
\item $a \mid b \wedge b \mid c \rightarrow a | c$
\item $c \mid a \wedge c \mid b \rightarrow c | (ma + nb)$
\end{itemize}

An integer $n$ is said to have \textit{odd} parity when $n \nmid 2$, and otherwise \textit{even} parity when $n \mid 2$. Often times an odd number can be written in the form $n = 2k + 1$ for some integer $k$, and $n = 2k$ for even numbers.

A \textit{prime} is an integer greater than 1 thatis divisible by no positive integers other than $1$ and itself. Otherwise, a number is said to be a \textit{composite}. Note that by this definition. The number $2$ is a prime. Often times there will be indications to exclude $2$ from primes by specifying odd primes.

Here are some facts about prime numbers, where $n \in \mathbb{Z}$:

\begin{itemize}
\item when $n > 1$, $p \mid n$ for some prime $p \leq n$
\item There are infinitely many primes
\item If $n$ is composite, then $p \mid n$ for some prime $p \leq \sqrt{n}$
\end{itemize}

The function $\pi(x)$ where $x$ is a positive real number denotes the number of primes not exceeding $x$.

\subsubsection*{Dirichlet's Theorem in Arithmetic Progressons}

Suppose that $a, b \in \mathbb{N}$ where $(a, b) = 1$. Then the arithmetic progression $an+b, n \in \mathbb{N}$ contains infinitely many primes.

\section{Greatest Common Divisors}

The \textit{greatest common divisor} (GCD) of $a, b \in \mathbb{Z}$ is the largest divisor $d$ such that $d \mid a$ and $d \mid b$. The GCD of $a$ and $b$ are often written as $\gcd(a, b)$ or $(a, b)$.

One way to describe the GCD is that a positive integer $d$ is a GCD iff:

\begin{itemize}
\item $d \mid a$ and $d \mid b$
\item if $c \in \mathbb{Z}$ s.t. $c \mid a$ and $c \mid b$, then $c \mid d$
\end{itemize}

Some facts about GCD's:

\begin{itemize}
\item Two integers $a, b \in \mathbb{Z}$ are said to be \textit{relatively prime} if $(a, b) = 1$
\item Suppose $d = ( a, b )$, then $(\frac{a}{d},\frac{b}{d}) = 1$
\item A fraction $\frac{p}{q}$ is in lowest terms when $(p, q) = 1$
\item suppose $(a, b) = 1$ and $a | bc$, then $a | c$
\item The notion of GCD's applies to multiple values too, suppose $a_1,a_2, \cdots, a_n \in \mathbb{Z}$, then \[(a_1, a_2, \cdots, a_n) = (a_1, a_2, \cdots, (a_{n-1}, a_n))\]
Note that simply with the above definition, $(a_1, a_2, \cdots, a_n) = 1$ means that these numbers are \textit{mutually relatively prime}. A stronger statement is \textit{pairwise relatively prime}, where for any pair of the numbers $a_i$ and $a_j$ with $i \neq j$, $(a_i, a_j) = 1$.
\end{itemize}

\subsubsection*{Bezout's Theorem}

If $a, b \in \mathbb{Z}$, then $\exists m,n \in \mathbb{Z}$ s.t. \[ma + nb = (a, b)\] Where $m,n$ are denoted the bezout coefficients to $a, b$. Furthermore, $(a,b) = 1 \Leftrightarrow ma +nb = 1$.

The set of linear combinations of $a,b$ is the set of integer multiples of $(a, b)$.

\subsubsection*{Euclidean Algorithm (+Backtracking)}

The Euclidean Algorithm computes $(a,b)$. It proceeds by heavily using a property of GCD's:

Let $b,q,r \in \mathbb{Z}$, \[(bq + r, b) = (b, r)\]

Suppose $a = bq+r$ (by the division algorithm), we have $(a,b) = (b, r)$. As $0 \leq r < b$, the RHS will always be in lower terms: each time we apply this property, we will get smaller computations to carry out until a base case of $(b, 0)$ is reached, in which case $(b, 0) = b$.

Once the series of division algorithms are applied, the results can be used in reverse with substitution to find the bezout coefficients.

As a side note, suppose we have $f_{n+1}, f_{n+2}$ be successive terms of the Fibonacci Sequence with $n > 1$, then the Euclidean algorithm takes exactly $n$ divisions to show that $(f_{n+1}, f_{n+2})= 1$.

\section{Continued Fraction Expansions}

Given the sequence $a_0, a_1, a_2, \cdots$ (may be infinite), a continued fraction is a fraction of the following form

\[a_0 + \frac{1}{a_1 + \frac{1}{a_2 + \cdots}}\]

A fraction of this form can also be written as $[a_0,a_1, a_2, \cdots]$. If the sequence is infinite and has repeating elements, we denote one instance of the repeating values with a line drawn over it.

The numbers $a_1, a_2, \cdots, a_n$ are called \textit{partial quotients} of the continued fraction, and if all $a_i$'s are integers, the continued fraction is said to be a \textit{simple} continued fraction. We are only concerned with \textit{simple} continued fractions and unless otherwise stated, all continued fractions under discussion are simple.

\subsection*{Finite Continued Fractions}

For a rational number that can be expressed as $\frac{p}{q}$, where $p,q \in \mathbb{Z}$ and $p > q$. Its continued fraction expansion is finite. The way to generate the expansion is to consider the division algorithms applies when computing $(p, q)$. For each row, it will be of the form $a = q\cdot b + r$. We apply the following transformation:

\[a = bq + r \Rightarrow \frac{a}{b} = q + \frac{1}{\frac{b}{r}}\]

The $\frac{b}{r}$ term will correspond to the LHS on the next row, recall that the next row would be computing $b$ divided by $r$, and thus a substitution can be made. We can continuously apply these substitutions until a continued fraction is formed.

\subsection*{Infinite Continued Fractions}

For an irrational number $n$, its continued fraction expansion is computed in the following manner:

\[
\begin{aligned}
&\;\;\; \alpha_0 = n \\
a_i = [\alpha_i] &\;\;\; \alpha_{i+1} = (\alpha_i - a_i)^{-1}
\end{aligned}
\]

It's good to note that $a_i$ is the integral component to $\alpha_i$, and that $\alpha_i - a_i$ is the fractional component to $\alpha_i$.

\subsection*{Convergents}

Given a continued fraction \[C = [a_0;a_1,a_2,\cdots,a_n]\] and $C_k = [a_0;a_1,a_2,\cdots,a_k]$ with $0 < k \leq n$, $C_k$ is called the $k$th convergent of $C$. Given $C$, we can compute all of the convergents of $C$ as follows:

\[
\begin{aligned}
& p_0 = a_0 & q_0 &= 1 &\\
& p_1 = a_0a_1+1 & q_1 &= a_1 & C_1 = \frac{p_1}{q_1}\\
& p_k = a_kp_{k-1}+p_{k-2}& q_k &= a_kq_{k-1}+q_{k-2} & C_k = \frac{p_k}{q_k}
\end{aligned}
\]

\section{Linear Diophantine Equations}

A linear diophantine equation in two variables is an equation of the follow form \[ax+by=c\] where $a,b,c \in \mathbb{Z}$ and an integer solution $(x,y)$ is sought for. In general, an equation that is linear in the coefficients to the polynomial powers would be considered a diophantine equation. We will focus on linear diophantine equations in two variables. Much of the results we find here applies to diophantine equations of more than two variables.

Let $d = (a, b)$. If $d \nmid c$, then the diophantine equation in question has no integral solutions. If $d \mid c$, then there are infinitely many solutions (granted that there are no restrictions on the solution space). Indeed, once we have an initial solution $(x_0, y_0)$, the rest of the solutions will all be of the form

\[x = x_0 + \frac{b}{d}n,\;\;\;y = y_0 - \frac{a}{d}n \]

for any integer $n$. Notice the minus sign on $y$.

\subsection*{Systems of Linear Diophantine Equations}

When there is a system of linear diophantine equations (most commonly two equations in 3 variables), the ideal approach is to substitute one equation into another, to reduce the system into a single equation. From there, solve as intended and substitute back for a full solution.

\section{The Fundamental Theorem of Arithmetic}

Every positive integer greater than 1 is \textit{uniquely} expressible as a product of primes, with the prime factors in non decreasing order: \[n = \prod_{i} p_i^{a_i}\]

The sequence of $a_i$'s denote the exponents to each of the prime factors. If a prime factor $p_i \nmid =n$, then $a_i = 0$.

Using this prime factorization form, we can describe the GCD and LCM (least common multiple) as follows, letting $a = \prod_{i}p_i^{a_i}$ and $b = \prod_{i}p_i^{b_i}$:

\[(a,b) = \prod_{i}p_i^{\min(a_i,b_i)}\;\;[a,b] = \prod_{i}p_i^{\max(a_i, b_i)}\]

As $\max(a_i, b_i) + \min(a_i, b_i) = a_i + b_i$, we can see from above that $(a,b) \cdot [a,b] = ab$

\section{Congruences}


Let $m$ be a positive integer. If $a,b \in \mathbb{Z}$, we say that $a$ is \textit{congruent} to $b$ modulo $m$ if $m \mid (a - b)$. Furthermore, we can write $a \equiv b \pmod{m}$ iff $a = b + km$ for some integer $k$. Here are some properties about arithmetic in modulo $m$:

\begin{itemize}
\item $a+c \equiv b + c \pmod{m}$
\item $a - c \equiv b - c \pmod{m}$
\item $ab \equiv bc \pmod{m}$
\end{itemize}

The congruence relation over a modulus forms an \textit{equivalence class}, which satisfies the following properties:

\begin{itemize}
\item Reflexitivity: $a \equiv a \pmod{m}$
\item Symmetricity: $a \equiv b \pmod{m} \Leftrightarrow b \equiv a \pmod{m}$
\item Transitivity: $a \equiv b \pmod{m} \wedge b \equiv c \pmod{m} \Rightarrow a \equiv c \pmod{m}$
\end{itemize}

A number in a modulo number system itself is referred to as a residue. If $a, b$ are congruent mod $m$, then their residues $a\pmod{m}$ and $b\pmod{m}$ have to be the same as well.

A complete system of residues modulo $m$ is a set of integers such that every integer is congruent to exactly one integer of the set. Of which, the least non-negative residues modulo $m$ is the set \[\{0,1,\cdots,m -1\}\] When $m$ is odd, the absolute least residues modulo $m$ is the set \[\left\{-\frac{m-1}{2}, -\frac{m-3}{2}, \cdots, -1, 0, 1, \frac{m-3}{2}, \frac{m-1}{2}\right\}\]

If $a,b,c,m \in \mathbb{Z}^+$ with $d = (c, m)$ and $ac \equiv bc \pmod{m}$, then $a \equiv b \pmod{\frac{m}{d}}$. In the case that $(c, m) = 1$, we simply have $a \equiv b \pmod{m}$.

If we have $a \equiv b \pmod{m_i}$ for several values of $i$ and each $m_i$ are pairwise coprime, then $a \equiv b \pmod{\prod_im_i}$.

For a given integer $a \in \mathbb{Z}/m\mathbb{Z}$, $a$ is either \textit{invertible} or a \textit{zero-divisor}. We focus on the case of invertibility. For an invertible $a$, its \textit{inverse} is a quantity $a^{-1}$ such that $a^{-1}a \equiv 1 \pmod{m}$.

For a prime $p$ and $(a,p) = 1$, \[a\equiv a^{-1} \pmod{p} \Leftrightarrow a \equiv \pm 1 \pmod{p}\]

\section{Linear Congruences}

A linear congruence in one variable is of the form \[ax \equiv b \pmod{m}\] It can be similarly seen as a linear diophantine equation of the form $ax - my = b$. Analoguous to the study of linear diophantine equations, we can see that with $(a, m) = d$, if $d \mid b$, then $ax \equiv b \pmod{m}$ has $d$ incongruent solutions. The way to find these solutions follow from solutions of linear diophantines of the same kind.

In particular, if $(a,m) = 1$, then there is one unique solution, which can by found by $x \equiv a^{-1}b \pmod{m}$.

\section{Chinese Remainder Theorem}

Let $m_1, m_2, \cdots, m_r$ be pairwise coprime positive integers. Then the system of congruences

\[
\begin{aligned}
x &\equiv a_1 \pmod{m_1} \\
x &\equiv a_2 \pmod{m_2} \\
&\cdots \\
x &\equiv a_r \pmod{m_r} \\
\end{aligned}
\]

has a unique solution modulo $M = \prod_i^rm_i$. The solution will be a sum where each term will turn to 0 for all modulo's except one. That term will be focused on solving specifically that congruence. It's important to take a mod $M$ at the end.

If $a,b \in \mathbb{Z}^+$, then the least positive residue of $2^a - 1$ modulo $a^b - 1$ is $2^r - 1$, where $r$ is the least positive residue of $a \pmod{b}$.

If $a,b \in \mathbb{Z}^+$, then $(2^a - 1, 2^b - 1) = 2^{(a,b)} - 1$. With this we can also see that $(2^a - 1, 2^b - 1) = 1 \Leftrightarrow (a, b) = 1$.

Suppose we have a polynomial $f(x)$ and we want to solve the congruence $f(x) \equiv 0 \pmod{m}$. We can obtain its solution by splitting $m$ into its prime factors and create a system of congruences. From there, we can proceed by applying the chinese remainder theorem.

\section{Wilson's Theorem}

If $n \geq 2$, then $n$ is prime iff \[(n - 1)! \equiv -1 \pmod{n}\]

\section{Fermat's Little Theorem}

If $p$ is prime and $a$ is an integer with $p \nmid a$, then \[a^{p-1} \equiv 1 \pmod{p}\]

This theorem can take on multiple forms, such as \[a^{p} \equiv a \pmod{p}\] and for finding inverses, \[a^{p -2} \equiv a^{-1} \pmod{p}\]

\section{Pseudoprimes}

Let $b$ be a positive integer. If $n$ is a composite positive integer and $b^n \equiv b \pmod{n}$, then $n$ is called a pseudoprime to the base $b$. Note that this shows that the converse of FLT does not hold. Generally pseudoprimes are composites that pass a certain primality condition, such as the converse of FLT.

If $d, n$ are positive integes s.t. $d \mid n$, then $2^d - 1 \mid 2^n - 1$. There are also infinitely many pseudoprimes to the base $2$.

A composite number $n$ is said to be a \textit{Carmichael} number if \[b^{n-1} \equiv 1 \pmod{n}\] for any natural $b$ where $(b,n) = 1$. If $n$ is of the form $n = p_1\cdot p_k$, where each $p_i$ are distinct primes and $p_i - 1 \mid n - 1$ for any $i$, then $n$ is a Carmichael number.

\section{Euler Totient}

The function $\phi(n)$ is defined to be the number of positive integers not exceeding $n$, which are relatively prime to $n$. For a value $n$ with the prime power factorization $n = \prod_{i}p_i^{a_i}$, $\phi(n)$ has many properties:

\begin{itemize}
\item $\phi(n) = n \prod_{p \mid n} \left(1 - \frac{1}{p}\right)$
\item $\phi(n) = \prod_{i} p_i^{a_i - 1}(p_i - 1)$
\item For a prime $p$, $\phi(p) = p - 1$
\item For a prime $p$, integer $a > 0$, $\phi(p^a) = p^a - p^{a-1}$
\item $a \mid b \Rightarrow \phi(a) \mid \phi(b)$
\item $(m,n) = 1 \Rightarrow \phi(mn) = \phi(m)\phi(n)$
\item $\phi(n)$ is even when $n > 2$
\end{itemize}

To get point 1 on above, consider point 4 above and carry out the following manipulations:

\[
\begin{aligned}
\phi(n) &= \phi(\prod_{i}p_i^{a_i}) \\
&= \prod_i \phi(p_i^{a_i}) \\
&= \prod_i (p_i^{a_i} - p_i^{a_i - 1}) \\
&= \prod_i p_i^{a_i}( 1 - p_i^{-1}) \\
&= n \prod_i ( 1 - p_i^{-1}) \\
&= n \prod_{p \mid n} \left(1 - \frac{1}{p}\right)
\end{aligned}
\]

The Euler Totient is considered a \textit{multiplicative} function, which will be described in greter detail later on.

With knowledge of the Euler totient, we can present a more generalized version of FLT:

\subsection*{Euler's Theorem}

If $m$ is a positive integer and $a$ is an integer with $(a,m) = 1$, then \[a^{\phi(m)} \equiv 1 \pmod{m}\]

\section{Multiplicative Functions}

A \textit{arithmetic} function is a function that is defined for all positive integers. An arithmetic function is \textit{multiplicative} when if $(m,n) = 1$, $f(mn) = f(m)f(n)$. A \textit{totally-multplicative} function ignores the coprimality condition. One example of a multiplicative function is the Euler Totient function.

A multiplicative function is easy to compute when it can be broken down into calls with smaller arguments.

Let $f$ be an arithmetic function, then

\[F(n) = \sum_{d \mid n} f(d)\]

is called the  \textit{summatory} function of $f$. A function $f$ is multiplicative iff its summatory function $F$ is multiplicative as well.

Let $n$ be a positive integer, then $n = \sum_{d \mid n}\phi(d)$.

\subsection*{Sum and Number of divisors}

The sum of divisors function $\sigma$ is defined as the sum of all positive divisors of $n$, including $n$ itself. The number of divisors function $\tau$ is defined as the number of positive divisors of $n$, including $n$ itself:

\[\sigma(n) = \sum_{d \mid n} d \;\;\; \tau(n) = \sum_{d \mid n} 1\]

Both $\sigma$ and $\tau$ are multiplicative.

Let $p$ be a prime and $a$ be a positive integers, we have

\[\sigma(p^a) = p^0 + p^1 + \cdots + p^a = \frac{p^{a+1} - 1}{p-1}\;\;\tau(p^a) = a + 1\]

Generalizing to non-primes, if we have $n = \prod_{i}p_i^{a_i}$,

\[\sigma(n) = \prod_{i}\frac{p_1^{a_i+1} - 1}{p_i-1}\;\;\tau(n) = \prod_{i}(a_i + 1)\]

\section{Perfect and Mersenne Numbers}

A positive integer $n$ is said to be a \textit{perfect} number if $\sigma(n) = 2n$. At the moment we only know of even perfect numbers and there aren't any know odd perfect numbers. An even number is perfect iff \[n = 2^{m-1}(2^m - 1)\] where $m \geq 2$ is an integer and $2^m - 1$ is prime.

If $m$ is a positive integer, then $M_m = 2^m - 1$ is called the $m$th \textit{Mersenne Number}. If $p$ is prime and $M_p = 2^p - 1$ is also prime, then $M_p$ is called a \textit{Mersenne Prime}. If $m$ is a positive integer and $M_m$ is prime, then $m$ must be prime.

If $p$ is an odd prime, then any divisor of $M_p$ is of the form $2kp+1$, where $k$ is a positive integer.

\section{Mobius Inversion}

To invert a summatory function, we can proceed as follows:

\[f(n) = \sum_{d \mid n} \mu(d)F(n/d)\]

where $\mu$ is a multiplicative Mobius function defined as follows:

\[
\mu(n) = \begin{cases}
1 & \text{if } n=1; \\
(-1)^r & \text{if } n = p_1p_2\cdots p_r, \text{where } p_i \text{ are primes;}\\
0 & \text{otherwise.}
\end{cases}
\]

The summatory function of $\mu$ satisfies

\[F(n) = \sum_{d\mid n} \mu(d) = \begin{cases}
1 & \text{if } n = 1; \\
0 & \text{if } n > 1.
\end{cases}
\]

Because $\sigma$ and $\tau$ are summatory functions to $f(n)=n$ and $f(n) = 1$, we can apply mobius inversion to get

\[\sum_{d \mid n} \mu\left(\frac{n}{d}\right)\sigma(d) = n,\;\;\; \sum_{d \mid n} \mu\left(\frac{n}{d}\right)\tau(d) = 1\]

\section{RSA Cryptography}

RSA is a public key encryption/decryption scheme that relies on the difficulty in factoring large numbers. The setup involves the following:

\begin{itemize}
\item Let $n = pq$, with $p, q$ being distinct primes
\item Encryption exponent $e \in \mathbb{Z}$ s.t. $(e, \phi(n)) =(e, \phi(pq)) = (e, \phi(p)\phi(q)) = (e, (p-1)(q-1))= 1$
\item Decryption exponent $d \in \mathbb{Z}$ s.t. $de \equiv 1 \pmod{\phi(n)}$
\end{itemize}

Letting $P$ be the plaintext and $C$ be the ciphertext, we define our encryption function $E$ and decryption function $D$ as follows:

\[ C = E(P) = P^e \pmod{n} \]
\[ P = D(C) = C^d \pmod{n} \]

Taking a closer look at how $D$ works,

\[
\begin{aligned}
D(C) &\equiv D(E(P)) &\pmod{n} \\
&\equiv (P^e)^d \equiv P^{ed} &\pmod{n}\\
&\equiv P^{1 + k\phi(n)} &\pmod{n}\\
&\equiv P(P^{\phi(n)})^k &\pmod{n}\\
&\equiv P &\pmod{n}\\
\end{aligned}
\]

Suppose there are two parties $A$, $B$, and $A$ wants to send $B$ a message. $B$ will start with generating two primes $p, q$ to compute $n$, $\phi(n) = (p-1)(q-1)$, and exponents $e$ and $d$. $B$ can now make $n, e$ public. Given $n, e$, $A$ can compute $C = E(P) = P^e \pmod{n}$ to send to $B$.

The security in this cryptosystem lies in the fact that given $e$ and $n$, it is difficult factoring $n$ to determine $d$.

One important fact to note is that if $\phi(n)$ was made public, it is easy to factor $n$ based off of that:

\[
\begin{aligned}
\phi(n) &= \phi(pq) \\
&= \phi(p)\phi(q) \\
&= (p-1)(q-1) \\
&= pq - (p+q) + 1 \\
\end{aligned}
\]

Therefore, $p+q$ is a known quantity. As an aside we observe that $(p-q)^2 = (p+q)^2 - 4pq = (p+q)^2 - 4n$. With $p - q = \sqrt{(p+q)^2 - 4n}$, so we can know $p + q$ and $p + q$. Therefore, $p, q$ can be obtained.

\section{Order of an integer}

Let $a, n in \mathbb{Z}$ with $(a,n) = 1$, $a \neq 0$, $n > 0$. The least positive $x \in \mathbb{Z}$ s.t. $a^x \equiv 1 \pmod{n}$ is called the \textit{order} of $a$ modulo $n$, denoted as $\text{ord}_na$. In other words

\[a^{\text{ord}_na} \equiv 1 \pmod{n}\]

A positive integer $x$ is a solution to $a^x \equiv 1 \pmod{n}$ iff $\text{ord}_na \mid x$.

It follows from Euler's Theorem that for any $a$ that is coprime to the modulo $n$, $\text{ord}_na \mid \phi(n)$.

If $(a,n) = 1$, $a^i \equiv a^j \pmod{n}$ iff $i \equiv j \pmod{\text{ord}_na}$.

\subsection*{Primitive Roots}

If $r,n \in \mathbb{Z}$ with $(r,n) = 1$, and if $\text{ord}_nr = \phi(n)$, then $r$ is called a \textit{primitive root} of $n$. Every prime has a primitive root.

If $(r,n) = 1$ and $r$ is a primitive root of $n$, then the integers $r^1,r^2,\cdots,r^{\phi(n)}$ form a reduced residue system modulo $n$. We can see that when $r$ is a primitive root and $(a,n) = 1$, $a \equiv r^x \pmod{n}$ for some $x$.

Suppose we have an exponent $u \in \mathbb{Z}^+$, then \[\text{ord}_n(a^u) = \frac{\text{ord}_na}{(\text{ord}_na, u)}\]

We can see above that if $a$ is a primitive root, then $a^u$ is a primitive root iff $(\text{ord}_na, u) = (\phi(n),u) = 1$.

If a positive integer $n$ has a primitive root, then it has a total of $\phi(\phi(n))$ incongruent primitive roots.

\section{Discrete Logarithms}

When $(a,n) = 1$ and $r$ is a primitive root, the $x$ in $a \equiv r^x \pmod{n}$ is called the \textit{index} or \textit{discrete logarithm} of $a$ to base $r$, denoted as $\text{ind}_ra$. Note that the modulus $n$ is absent from the notation. We observe a few properties of discrete logarithms, note the typical continuous logarithm rules apply:

\begin{itemize}
\item $r^{\text{ind}_ra}\equiv a \pmod{n}$
\item $\text{ind}_r1 \equiv 0 \pmod{\phi(n)}$
\item $\text{ind}_r(ab) \equiv \text{ind}_ra + \text{ind}_rb \pmod{\phi(n)}$
\item $\text{ind}_ra^k \equiv k \cdot \text{ind}_ra \pmod{\phi(n)}$, $k \in \mathbb{Z}^+$
\end{itemize}

Suppose $n$ has a primitive root, if we have an integer $a$ s.t. $(a,n) = 1$, then $x^k \equiv a \pmod{n}$ has a solution. Similarly, $a^{\frac{\phi(n)}{d}} \equiv 1 \pmod{n}$ where $d = (k, \phi(n))$. Furthermore, if there are solution of $x^k \equiv a \pmod{n}$, then there are exactly $d$ incongruent solutions mod $n$. A way to interpret such results is that if an integer $a$ is a $k$th power residue iff $a^{\frac{p-1}{(k, p-1)}} \equiv 1 \pmod{p}$.

\section{Distribution Of Primes}

A function $\pi(n)$ counts the number of primes up to $n$. It is importantly used in the following theorem:

\subsubsection*{Prime Number Theorem}

\[\lim_{x \rightarrow \infty} \frac{\pi(x)}{\frac{x}{\ln{x}}} = 1\]

In other words, $\frac{x}{\ln{x}}$ and $\pi(x)$ are asymptotic. This is often written as $\pi(x) \sim \frac{x}{\ln{x}}$. When $a(x) \sim b(x)$, $\lim_{x \rightarrow \infty}\frac{a(x)}{b(x)} = 1$.

Let $p_n$ be the $n$th prime, then $p_n \sim n\log{n}$.

For any $n > 0$, there are at least $n$ consecutive composite positive integers.

\newpage

\section{Infinimality of primes of the form $4k+3$}

Before the actual proof can take place, we note that primes can only either be of the form $4k+1$ or $4k+3$, which are the odd numbers in mod 4.

Also we note that if two numbers $a, b$ are both of the form $4k +1$, then $ab$ will also be of the form $4k + 1$ (proof omitted for brevity).

To begin the actual proof, suppose we have a finite amount of primes of the form $4k+3$, Namely, all in the sequence $\{p_i\}$ starting with $p_0 = 3$. We construct the following: \[Q = 4\prod_{i=1}{p_i} + 3\]Note that $p_0 = 3$ is omitted from the expression.

Clearly, $Q$ is a composite value of the form $4k+3$ by construction. It needs to have a factor that is of the form $4k+3$, as we have established that multiplication between numbers of the form $4k+1$ is closed. Therefore, for some $i$, $p_i \mid Q$. However, none of the $p_i$'s can divide $Q$ by construction. Therefore, $Q$ is another prime of the form $4k+3$, a contradiction to the assumption that there are only a finite amount of them.

\end{multicols*}
\end{document}
